\chapter{Seminar\+Lite\+RT }
\hypertarget{md__r_e_a_d_m_e}{}\label{md__r_e_a_d_m_e}\index{SeminarLiteRT@{SeminarLiteRT}}
\label{md__r_e_a_d_m_e_autotoc_md0}%
\Hypertarget{md__r_e_a_d_m_e_autotoc_md0}%


Vertiefungsseminar zum Thema Lite\+Rt

Testweise :\+Beispiel Anw von Google zum laufen bringen App 1 Lite\+RT Modell in eigener App einbinden-\/\texorpdfstring{$>$}{>} selbes wie im Test. Classifier selbst erstellt

App 2 Fertiges (einfaches) Modell in Lite\+RT Model umwandeln und zum laufen bringen

App 3 Fertiges anspruchsvolleres Modell umwandeln und zum laufen bringen

App4 Einfaches Modell (Google Teachable machine ) selber erstellen und laufen lassen

\href{https://www.kaggle.com/models?tfhub-redirect=true}{\texttt{https:\+/\+/\+www.\+kaggle.\+com/\+models?tfhub-\/redirect=true}}

Lite\+RT-\/\+LM runtime nochmal eine Andere Umgebung -\/\texorpdfstring{$>$}{>} komplett anderes prinzip 